%% =============================
%% GENERAL SETTINGS
%% =============================
\documentclass[12pt]{article}

\usepackage[margin=0.7in]{geometry}
\usepackage {graphics}
\usepackage{xltxtra} 
\usepackage{xgreek} 

\setmainfont[Mapping=tex-text]{Tahoma} 
\setlength{\parindent}{0cm} 				%% No paragraph indent

%% =============================
%% DOCUMENT PROPERTIES
%% =============================
\title{Easy!Appointments - Vision Document}
\author{Αλέξανδρος Τσελεγγιδης}
\date{Νοέμβριος 2012}

%% =============================
%% DOCUMENT CONTENT
%% =============================
\begin{document}
\maketitle 
\pagebreak

%% ΣΚΟΠΟΣ ΤΟΥ ΕΡΓΟΥ	
\section{Σκοπός του Έργου}
Σκοπός του Easy!Appointments ειναι να αναπτυχθεί ένα web σύστημα κρατήσεων ραντεβού (για οποιαδήποτε υπηρεσία ή γραφείο) το οποίο να μπορεί να παραμετροποιείται καταλλήλως και να συγχρονίζεται με το Google Calendar. Στόχος είναι να παράγεται κώδικας που να είναι συμβατός με οποιοδηποτε διακοσμητή AMP.

%% ΠΙΘΑΝΟΙ ΠΕΛΑΤΕΣ
\section{Πιθανοί Πελάτες}
Οποιαδήποτε εταιρεία ή επιχείρηση λειτουργεί με ραντεβού σε γραφείο ή κατάστημα μπορεί να χρησιμοποιήσει την εφαρμογή για να διαχειρίζεται τα ραντεβού ηλεκτρονικά και μέσω του internet. Αυτήν την στιγμή δεν υπάρχουν αντιστοιχα προγρ

%% ΚΥΡΙΟΙ ΠΑΡΑΓΟΝΤΕΣ
\section{Κύριοι Παράγοντες}
\begin{enumerate}
\item{Αξιοπιστια}
\item{Ταχύτητα}
\item{Ευχρηστία}
\item{Συνεργασία μεταξύ διαφόρων υπαλλήλων}
\end{enumerate}

%% ΚΥΡΙΑ ΧΑΡΑΚΤΗΡΙΣΤΙΚΑ
\section{Χαρακτηριστικά και Τεχνολογία}
\begin{enumerate}
\item{Χρήση της βιβλιοθήκης Google Calendar API}
\item{Χρήση της εφαρμογής Google Calendar Sync (παροχή αυτόματων ενημερώσεων)}
\item{Εύχρηστο και όμορφο περιβάλλον}
\item{Ασφαλής λειτουργία της εφαρμογής (χωρίς απώλειες δεδομένων)}
\item{Υλοποίηση με PHP, MySQL, jQuery, CodeIgniter Framework}
\end{enumerate}

%% ΑΛΛΟΙ ΣΗΜΑΝΤΙΚΟΙ ΠΑΡΑΓΟΝΤΕΣ
\section{Άλλοι Παράγοντες}
\begin{enumerate}
\item{Η εφαρμογή δεν πρέπει να "κρασάρει"}
\item{Δυνατότητα ενημέρωσης με SMS των χρηστών για αλλαγές στο εβδομαδιαίο πλάνο τους}
\end{enumerate}

%% ΟΙΚΟΝΟΜΙΚΟΙ ΠΑΡΑΓΟΝΤΕΣ
\section{Οικονομικοί Παράγοντες}
Πρέπει να προωθηθεί σε όλα τα site που ασχολούνται με τα CMS, το ελεύθερο λογισμικό και τα επιχειρηματικά βοηθήματα.
\\[0.3cm]
Η εφαρμογή θα διατίθεται δωρεάν μέσω της άδειας GNU/GPL και τα έσοδα θα προέρχονται από διαφημίσεις, δωρεές, εκπαίδευση και την συντήρηση.

\end{document}