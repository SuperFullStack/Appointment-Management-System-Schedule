%% =============================
%% GENERAL SETTINGS
%% =============================
\documentclass[12pt]{article}

\usepackage [margin=2.5cm]{geometry}
\usepackage {graphics}
\usepackage {xltxtra} 
\usepackage {xgreek} 
\usepackage {color}
\usepackage {hyperref}
\hypersetup {colorlinks}


\setmainfont[Mapping=tex-text]{Tahoma} 
\setlength{\parindent}{0cm} 				%% No paragraph indent

\definecolor{darkred}{rgb}{0.5,0,0}
\definecolor{darkgreen}{rgb}{0,0.5,0}
\definecolor{darkblue}{rgb}{0,0,0.5}

\hypersetup{ colorlinks,
linkcolor=darkblue,
filecolor=darkgreen,
urlcolor=darkblue,
citecolor=darkred }

%% =============================
%% DOCUMENT PROPERTIES
%% =============================
\title{{\Huge {\bf Easy!Appointments}} \\[0.3cm] Σενάριο Χρήσης}
\author{Αλέξανδρος Τσελεγγίδης}
\date{Νοέμβριος 2012}

%% =============================
%% DOCUMENT CONTENT
%% =============================
\begin{document}
\maketitle 
\thispagestyle{empty} %% Απομάκρυνση page number από την πρώτη σελίδα
\pagebreak

%% ΣΕΝΑΡΙΟ ΧΡΗΣΗΣ ΔΙΑΧΕΙΡΙΣΤΗ
\section{Σενάριο Χρήσης Διαχειριστή}
Ο διαχειριστής της εφαρμογής λαμβάνει μια ειδοποίηση από την εφαρμογή (email ή sms) ότι έχει γίνει μια κράτηση για ραντεβού. Βλέποντας τα στοιχεία της κράτησης και την ημερομηνία αποφασίζει ότι δεν θα μπορέσει να είναι εκείνη την στιγμή διαθέσιμος, οπότε συνδέεται στην εφαρμογή και προτείνει κάποια άλλη ημερομηνία. Μετά από λίγο έρχεται μια άλλη ειδοποίηση στην οποία ο πελάτης έχει αποδεχτεί την αλλαγή και έτσι το ραντεβού κατοχυρώνεται. Αμέσως μετά ο διαχειριστής πηγαίνει στο πρόγραμμά του και ενημερώνει την χρονική στιγμή στην οποία δεν θα είναι διαθέσιμος, έτσι ώστε να μην μπορούν πλέον οι πελάτες να κάνουν κρατήσεις σε εκείνη την χρονική περίοδο.

%% ΣΕΝΑΡΙΟ ΧΡΗΣΗΣ ΠΕΛΑΤΗ
\section{Σενάριο Χρήσης Πελάτη}
Ο πελάτης ενδιαφέρεται να κλείσει ραντεβού στην επιχείρηση για μια συγκεκριμένη υπηρεσία. Πηγαίνει στην σελίδα της επιχείρησης και βλέπει το πλάνο, αφού έχει επιλέξει ποια υπηρεσία και ποιόν υπάλληλο προτιμάει. Στην συνέχεια επιλέγει μια χρονική στιγμή που τον βολεύει και την κατοχυρώνει. Για να ολοκληρωθεί η διαδικασία θα χρειαστεί να απαντήσει σε ένα επιβεβαιωτικό mail που θα του έρθει. Από την στιγμή αυτήν και μετά το ραντεβού έχει κατοχυρωθεί και ο πελάτης μπορεί να ενημερωθεί ανά πάσα στιγμή για την κατάστασή του.

\end{document}