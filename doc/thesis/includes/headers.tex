\documentclass[oneside, 12pt]{book}

%% ============================================================================
%% ΟΡΙΣΜΟΣ ΤΩΝ PACKAGES ΠΟΥ ΘΑ ΧΡΗΣΙΜΟΠΟΙΗΘΟΥΝ
%% ============================================================================
\usepackage{thesis}
\usepackage{tabularx} 
\usepackage{epsfig}
\usepackage{float}
\usepackage{listings}
\usepackage{hyperref}
\usepackage{color}
\usepackage{xcolor}

%% ============================================================================
%% ΡΥΘΜΙΣΗ ΤΩΝ HYPERLINKS
%% ============================================================================
\hypersetup {colorlinks}
\definecolor{darkred}{rgb}{0.5,0,0}
\definecolor{darkgreen}{rgb}{0,0.5,0}
\definecolor{darkblue}{rgb}{0,0,0.5}
\hypersetup{ 
	colorlinks,
	linkcolor=darkblue,
	filecolor=darkgreen,
	urlcolor=darkblue,
	citecolor=darkred 
}

%% ============================================================================
%% PHP LISTINGS STYLE
%% http://tex.stackexchange.com/a/54687
%% http://en.wikibooks.org/wiki/LaTeX/Source_Code_Listings
%% http://en.wikibooks.org/wiki/LaTeX/Colors
%% ============================================================================
\definecolor{dkgreen}{rgb}{0,.6,0}
\definecolor{dkblue}{rgb}{0,0,.6}
\definecolor{dkyellow}{cmyk}{0,0,.8,.3}
\definecolor{ltgrey}{RGB}{240,240,240}
\lstset{
  language        = php,
  basicstyle      = \footnotesize\ttfamily,
  keywordstyle    = \color{dkblue},
  stringstyle     = \color{red},
  identifierstyle = \color{dkgreen},
  commentstyle    = \color{gray},
  emph            =[1]{php},
  emphstyle       =[1]\color{black},
  emph            =[2]{if,and,or,else,public,function,try,catch,return},
  emphstyle       =[2]\color{dkyellow},
  numbers		  = left,
  tabsize		  = 2,
  backgroundcolor = \color{ltgrey},
  extendedchars   = true,
  showspaces	  = false,
  showstringspaces= false}

%% ============================================================================
%% ΤΑ ΠΑΡΑΚΑΤΩ ΕΙΝΑΙ ΥΠΟΧΡΕΩΤΙΚΑ
%% ============================================================================
\renewcommand{\thesistitle}{Δημιουργία διαδικτυακού συστήματος συναντήσεων (appointments) με χρήση Google Calendar PHP API}
\renewcommand{\thesisauthor}{Αλέξανδρος Τσελεγγίδης (2503)}
\renewcommand{\thesisauthorabbrv}{Α. Τσελεγγίδης}
\renewcommand{\thesisauthorinitials}{ΑΤ}
\renewcommand{\thesissupervisor}{Δρ. Νικόλαος Πεταλίδης, Επιστημονικός Συνεργάτης}
\renewcommand{\thesismonth}{Νοέμβριος}
\renewcommand{\thesisyear}{2013}

%% ΒΙΒΛΙΟΓΡΑΦΙΑ
\addbibresource{thesis.bib}
