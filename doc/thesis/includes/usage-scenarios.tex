%% ΣΕΝΑΡΙΟ ΧΡΗΣΗΣ
%% Σε αυτό το κεφάλαιο γίνεται περιγραφή ενός σεναρίου χρήσης του συστήματος
%% για κάθε έναν από τους ρόλους των χρηστών της εφαρμογής.

\chapter{Σενάρια Χρήσης}
Το κεφάλαιο αυτό έχει ως στόχο να δώσει μια τυπική περιγραφή της χρήσης της εφαρμογής, για όλους τους διαθέσιμους ρόλους των χρηστών της, έτσι ώστε να γίνει περισσότερο κατανοητός ο τρόπος με τον οποίον θα λειτουργεί το σύστημα κρατήσεων ραντεβού.

%% ΣΕΝΑΡΙΟ ΧΡΗΣΗΣ ΔΙΑΧΕΙΡΙΣΤΗ
\section{Σενάριο Χρήσης Διαχειριστή}
Μετά από αρκετό καιρό χρήσης του Easy!Appointments η εταιρεία προσθέτει μια νέα υπηρεσία στο ενεργητικό της και για τον σκοπό αυτό ανοίγει ένα νέο τμήμα υπαλλήλων. Ο διαχειριστής του συστήματος πρέπει να ενημερώσει την εφαρμογή και να προσθέσει την νέα υπηρεσία, καθώς και τους νέους πάροχους υπηρεσιών, έτσι ώστε να μπορούν οι πελάτες να κλείνουν ραντεβού μαζί τους από εδώ και πέρα. Εφόσον γίνει αυτό, οι πελάτες θα μπορούν να επιλέξουν τις αντίστοιχες εγγραφές από την φόρμα κράτησης ραντεβού.

%% ΣΕΝΑΡΙΟ ΧΡΗΣΗ ΠΑΡΟΧΟΥ ΥΠΗΡΕΣΙΩΝ
\section{Σενάριο Χρήσης Πάροχου Υπηρεσιών}
Ο πάροχος υπηρεσιών της εφαρμογής λαμβάνει μια ειδοποίηση από την εφαρμογή (email) ότι έχει γίνει μια κράτηση για ραντεβού. Βλέποντας τα στοιχεία της κράτησης και την ημερομηνία αποφασίζει ότι δεν θα μπορέσει να είναι εκείνη την στιγμή διαθέσιμος, οπότε συνδέεται στην εφαρμογή και αλλάζει την ημερομηνία του ραντεβού. Αμέσως μετά πηγαίνει στο πρόγραμμά του και ενημερώνει την χρονική στιγμή στην οποία δεν θα είναι διαθέσιμος, έτσι ώστε να μην μπορούν πλέον οι πελάτες να κάνουν κρατήσεις σε εκείνη την χρονική περίοδο. Στην συνέχεια αποστέλλεται ειδοποίηση στον πελάτη και αυτός μπορεί να κρίνει αν τον βολεύει η νέα ημερομηνία. Αν όχι θα πρέπει να ακυρώσει το ραντεβού και να το ξανά-προσθέσει σε κάποια άλλη χρονική στιγμή. 

%% ΣΕΝΑΡΙΟ ΧΡΗΣΗΣ ΠΕΛΑΤΗ
\section{Σενάριο Χρήσης Πελάτη}
Ο πελάτης ενδιαφέρεται να κλείσει ραντεβού στην επιχείρηση για μια συγκεκριμένη υπηρεσία. Πηγαίνει στην σελίδα της επιχείρησης και βλέπει το πλάνο, αφού έχει επιλέξει ποια υπηρεσία και ποιόν υπάλληλο προτιμάει. Στην συνέχεια επιλέγει μια χρονική στιγμή που τον βολεύει και την κατοχυρώνει. Για να ολοκληρωθεί η διαδικασία θα χρειαστεί να απαντήσει σε ένα επιβεβαιωτικό email που θα του έρθει. Από την στιγμή αυτήν και μετά το ραντεβού έχει κατοχυρωθεί και ο πελάτης μπορεί να ενημερωθεί ανά πάσα στιγμή για την κατάστασή του.

\section{Σενάριο Χρήσης Γραμματέας}
Ένας από τους πάροχους υπηρεσίας έχει κλειστεί εντελώς από ραντεβού και δεν μπορεί να δεχτεί άλλα για αυτήν την εβδομάδα. Ένας άλλος πάροχος προσφέρεται να βοηθήσει και έτσι κάποια ραντεβού πρέπει να μεταφερθούν στο ημερολογιακό πλάνο του δεύτερου πάροχου. Την διαδικασία αυτήν θα πρέπει να την αναλάβει η γραμματεία γιατί όλοι οι άλλοι είναι πολύ απασχολημένοι με το να εξυπηρετήσουν τους πελάτες τους.