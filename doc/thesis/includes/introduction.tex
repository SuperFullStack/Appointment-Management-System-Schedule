%% ΕΙΣΑΓΩΓΙΚΟ ΚΕΦΑΛΑΙΟ
%% Το μέρος του εγγράφου αυτού περιέχει τα πρώτα μέρη της διπλωματικής
%% (πρωτοσέλιδο, περιεχόμενα, πρόλογος κτλ) καθώς και ένα κεφάλαιο που
%% αναλύει και περιγράφει το πρόβλημα που επιλύει το σύστημα που αναπτήχθηκε.

\Titlepage
\Declarationpage

\begin{Abstract}
Η εργασία αυτή πραγματεύεται την υλοποίηση ενός διαδικτυακού συστήματος κρατήσεων ραντεβού για επιχειρήσεις, με πλήρη περιβάλλον διαχείρισης και την δυνατότητα συγχρονισμού των ραντεβού με το Google Calendar API. Στόχος είναι να κατασκευαστεί ένα ευέλικτο σύστημα, το οποίο να είναι σε θέση να εξυπηρετήσει τις ανάγκες οποιασδήποτε επιχείρησης, βελτιώνοντας έτσι την μηχανογράφηση και την οργάνωση της και κατ' επέκταση την απόδοση της. Όλα αυτά σαφώς συντελούν στην μείωση του κόστους λειτουργίας, κάτι το οποίο είναι πολύ σημαντικό. Για την υλοποίηση της εφαρμογής επιλέχθηκε η γλώσσα προγραμματισμού PHP και Javascript καθώς και κάποιες εξωτερικές βιβλιοθήκες κώδικα, οι οποίες φάνηκαν πολύ χρήσιμες κατά την υλοποίηση του συστήματος.
\end{Abstract}

\tableofcontents

\begin{Definitions}
Στο έγγραφο αυτό υπάρχουν κάποιες έννοιες οι οποίες χρησιμοποιούνται 
σε διάφορα σημεία. Παρακάτω παρατίθενται οι περιγραφές τους.
\begin{description}
\item [Διαχειριστής] Ο διαχειριστής του συστήματος είναι ο χρήστης ο οποίος έχει όλα τα δικαιώματα αλλαγών και ρυθμίσεων του Easy!Appointments. Μπορεί να ορίσει νέες υπηρεσίες και πάροχους υπηρεσίας, να ρυθμίσει το σύστημα ειδοποιήσεων και να εκτελέσει όλες τις δυνατές διαδικασίες διαχείρισης των δεδομένων.

\item [Πάροχος Υπηρεσίας] Ο πάροχος υπηρεσίας είναι η οντότητα που εξυπηρετεί μια ή περισσότερες υπηρεσίες. Μπορεί να αντιπροσωπεύει ένα άτομο ή μια ομάδα ατόμων. Σε κάθε περίπτωση όμως διαχειρίζεται από έναν χρήστη του συστήματος.

\item [Πελάτης] Ο πελάτης αφού δει τις διαθέσιμες ημερομηνίες και ώρες για τις επιλεγμένες υπηρεσίες και παρόχους, μπορεί να κλείνει ραντεβού με την επιχείρηση. Αν γίνει οποιαδήποτε αλλαγή σε κάποιο ραντεβού του πελάτη τότε αυτός θα ενημερωθεί με σχετικό email.

\item [Γραμματέας] Ο γραμματέας είναι ένας χρήστης ο οποίος μπορεί να διαχειριστεί τα ραντεβού και τους πελάτες του συστήματος για συγκεκριμένους πάροχους υπηρεσιών. Το σε ποιους πάροχους αντιστοιχεί ο κάθε χρήστης γραμματέας ορίζεται από τον διαχειριστή στο περιβάλλον ρυθμίσεων της εφαρμογής.

\item [Πλάνο Πάροχου] Από την στιγμή που κλείνονται ραντεβού σε έναν πάροχο υπηρεσιών το ημερολογιακό του πλάνο αρχίζει να γεμίζει από χρονικά διαστήματα, τα οποία είναι δεσμευμένα και αντιπροσωπεύουν συναντήσεις με τους πελάτες. Εκτός αυτού υπάρχει και η δυνατότητα να τεθεί ένα ανενεργό χρονικό διάστημα, στο οποίο ο συγκεκριμένος πάροχος δεν θα είναι διαθέσιμος έτσι ώστε να μην μπορούν οι πελάτες να κλείνουν ραντεβού σε αυτό το διάστημα. Αυτό το πλάνο μπορεί να συγχρονιστεί με το Google Calendar έτσι ώστε να είναι προσβάσιμο και από άλλες υπηρεσίες. 

\item [Easy!Appointments] Με την ονομασία Easy!Appointments θα γίνεται αναφορά στο σύστημα που υλοποιήθηκε.
\end{description}
\end{Definitions}

%% =============================================
%% ΚΕΦΑΛΑΙΟ 1 - ΕΙΣΑΓΩΓΗ
%% =============================================
\chapter{Εισαγωγή}
\leftmark\rightmark
Το παρόν κεφάλαιο επεξηγεί τον σκοπό ανάπτυξης του Easy!Appointments καθώς και τις ανάγκες που καλύπτει σε μια επιχείρηση. Επιπρόσθετα αναφέρονται παρόμοια συστήματα και οι διαφορές που έχουν σε σχέση με την εφαρμογή.

\section {Ποια προβλήματα προσπαθεί να λύσει η εφαρμογή}
Οι επιχειρήσεις από την φύση τους χρειάζεται να έρχονται σε επαφή με τους πελάτες, για να μπορέσουν να τους εξυπηρετήσουν και έτσι να πάρουν την αμοιβή τους. Ανάλογα με την μορφή και το είδος της επιχείρησης, η επαφή αυτή διαφέρει. Για παράδειγμα κάποιες επιχειρήσεις έρχονται σε επαφή με περισσότερους πελάτες, άλλες με λιγότερους αλλά η εξυπηρέτηση μπορεί να είναι παθητική (πχ κατάστημα ηλεκτρονικών ειδών) και κάποιες απαιτούν ιδιαίτερη προσοχή στον πελάτη καθώς η εξυπηρέτησή του μπορεί να γίνει μόνο προσωπικώς, από κάποιον υπάλληλο ή επαγγελματία (πάροχος της υπηρεσίας). Η τελευταία κατηγορία περιέχει ένα μεγάλο εύρος επιχειρήσεων το οποίο για να οργανώσει και να διευκολύνει το πελατειακό κοινό του, λειτουργεί κανονίζοντας ραντεβού με τους ενδιαφερόμενους πελάτες.

Η κράτηση ενός ραντεβού είναι μια διαδικασία η οποία γίνεται συνήθως τηλεφωνικώς, είτε μετά από προσωπικό κανονισμό με κάποιον αρμόδιο. Η διαδικασία αυτή αποτελείται συνήθως από τα παρακάτω μέρη:  
\begin{enumerate}
\item Ο ενδιαφερόμενος πελάτης έρχεται σε επαφή με την επιχείρηση και ζητάει να κάνει κράτηση την επιθυμητή ημερομηνία και ώρα, για μια συγκεκριμένη υπηρεσία. 
\item Ο αρμόδιος υπάλληλος ψάχνει σε κάποιο ημερολόγιο ή αρχείο υπολογιστή τα ραντεβού για την συγκεκριμένη ημερομηνία και ανάλογα με την διαθεσιμότητα ανταποκρίνεται στον πελάτη.
\item Αν η συγκεκριμένη χρονική στιγμή δεν είναι διαθέσιμη θα χρειαστεί να γίνει μια αντιπρόταση από τον υπάλληλο ή ο πελάτης να βρει κάποια άλλη στιγμή που θα είναι αυτός διαθέσιμος.
\end{enumerate} 
Αν παρατηρήσουμε όμως την παραπάνω διαδικασία, θα δούμε πως έχει κάποια σημαντικά μειονεκτήματα, τα οποία μάλιστα συνεπάγονται την αύξηση του κόστους λειτουργίας μιας επιχείρησης και την μείωση της ποιότητας εξυπηρέτησης των πελατών. 

Η  ίδια η διαδικασία της κράτησης ενός ραντεβού με τον συγκεκριμένο τρόπο απαιτεί από μόνη της την ύπαρξη ενός υπαλλήλου, ο οποίος θα σπαταλάει αρκετό, αν όχι τον περισσότερο, από τον χρόνο του για να κάνει αυτήν την εργασία. Αυτό πρακτικά σημαίνει δέσμευση ανθρώπινων πόρων της επιχείρησης και συνεπάγεται στην αύξηση των εξόδων λειτουργίας. 

Επιπλέον, η ίδια η διαδικασία μπορεί να είναι χρονοβόρα και κουραστική για τους πελάτες, ειδικά στις περιπτώσεις όπου υπάρχει λίγο προσωπικό για να καλύψει μεγάλο κοινό (πχ νοσοκομεία). Στις περιπτώσεις αυτές οι πελάτες περιμένουν στην αναμονή για μεγάλο χρονικό διάστημα και μάλιστα πολλές φορές δεν πιάνουν γραμμή για να μπορέσουν να κρατήσουν κάποιο ραντεβού. Επίσης πρέπει να σημειωθεί ότι όταν ο πελάτης καταφέρει να κλείσει το ραντεβού του, συνήθως δεν έχει επιλογή για το πότε θα γίνει και απλώς ενημερώνεται για την ημερομηνία την οποία έχει ορίσει το προσωπικό, ανάλογα με τις εκάστοτε συνθήκες.  

Εκτός αυτών η εκτέλεση αυτής της διαδικασίας είναι αρκετά επιρρεπής στο να έχει ασαφές αποτελέσματα, με την έννοια του ότι δεν υπάρχει κάποιο κοινό σημείο αναφοράς για την συμφωνία που πραγματοποιείται μεταξύ της επιχείρησης και του πελάτη, έτσι ώστε να μπορεί να γίνει εξακρίβωση και επαλήθευση των ιδιοτήτων μιας κράτησης και από τις δύο πλευρές. Αυτό μπορεί να οδηγήσει σε προβλήματα με τους πελάτες, κάτι το οποίο δεν είναι επιθυμητό σε καμία περίπτωση.

Μικρό είναι το μέρος των πληροφοριών που καταγράφεται με το πέρας της κράτησης, καθώς τα μέσα που χρησιμοποιούνται δεν επιτρέπουν ή κάνουν δύσκολη και χρονοβόρα την αποθήκευση όλων των δεδομένων. Αυτό συντελεί στην επιπλέον μείωση της ποιότητας εξυπηρέτησης και της απόδοσης της επιχείρησης.

Τα δεδομένα αυτά διαχειρίζονται συνήθως δύσκολα. Ακόμα και στην περίπτωση που χρησιμοποιούνται ηλεκτρονικά μέσα για την αποθήκευση των κρατήσεων, η τροποποίηση ή ο έλεγχος μπορούν να είναι δύσκολες και χρονοβόρες διαδικασίες, οι οποίες εξαρτώνται κάθε φορά από το επίπεδο της οργάνωσης της επιχείρησης και τις τεχνολογίες που χρησιμοποιούνται. 

Επίσης τα δεδομένα αυτά δεν είναι προσβάσιμα από οποιονδήποτε ανά πάσα στιγμή, αλλά μόνο στον χώρο της επιχείρησης και μόνο από το άτομο το οποίο διαχειρίζεται τα ραντεβού. 

Τα παραπάνω προβλήματα διογκώνονται σημαντικά όταν πρόκειται για μεγάλες επιχειρήσεις και οργανισμούς, οι οποίοι εξυπηρετούν μεγάλο αριθμό πελατών.

\section {Γιατί είναι σημαντικά τα προβλήματα αυτά}
Οι απαιτήσεις και η ανταγωνιστικότητα που υπάρχει μεταξύ των επιχειρήσεων στην εποχή μας, απαιτεί την γρήγορη και άμεση διεκπεραίωση διεργασιών και την όσο το δυνατόν καλύτερη οργάνωση τους, για να μπορούν να παρέχουν υπηρεσίες υψηλού επιπέδου με το χαμηλότερο δυνατό κόστος και προσωπικό. Για να επιτύχουν τον σκοπό αυτό οι επιχειρήσεις πρέπει να επιλέξουν τα κατάλληλα εργαλεία οργάνωσης και εξυπηρέτησης των πελατών τους.

Βλέποντας τα προβλήματα που αναφέρθηκαν προηγουμένως είναι κατανοητό ότι με την χρήση της έως τώρα μεθόδου κράτησης ραντεβού, επέρχεται μείωση της ποιότητας και της απόδοσης της επιχείρησης. Αυτό σημαίνει ότι το επίπεδο εξυπηρέτησης είναι χαμηλότερο και έτσι η επιχείρηση αδυνατεί να είναι ανταγωνιστική προς τις άλλες καθώς γίνεται σπατάλη πόρων για την υλοποίηση αυτής της διαδικασίας.

Η μείωση αυτή επιφέρει αύξηση του κόστους λειτουργίας το οποίο αποτελεί ένα επιπρόσθετο εμπόδιο στην προσπάθεια για ανάπτυξη και επέκταση. Πολλές φορές μάλιστα αυτή η αύξηση του κόστους σε συνδιασμό με άλλους παράγοντες μπορεί να συντελέσουν στην μη βιωσιμότητα και το κλείσιμο της επιχείρησης, εφόσον αυτή δεν μπορεί να παράγει κέρδη.

Παρατηρούμε λοιπόν ότι η σημερινή οργάνωση των επιχειρήσεων που λειτουργούν με ραντεβού, θα μπορούσε να βελτιωθεί με την χρήση ενός ηλεκτρονικού συστήματος που θα επίλυε τα προαναφερθέντα προβλήματα και θα πρόσδιδε μεγαλύτερη ευκολία στην εξυπηρέτηση του κοινού. Η προτεινόμενη λύση αποσκοπεί στο να εκπληρώσει αυτά τα κενά και να εντάξει στο ενεργητικό της επιχείρησης ένα δυνατό εργαλείο οργάνωσης.

\section{Παρόμοιες λύσεις που υπάρχουν ήδη}
\subsection{Genbook}
Το Genbook είναι μια online υπηρεσία που προσφέρει στις επιχειρήσεις την δυνατότητα να εγγραφούν (πληρώνοντας το αντίτιμο) και να χρησιμοποιήσουν την εφαρμογή που τους επιτρέπει να διαχειρίζονται τα ραντεβού. Παρέχει αρκετά φιλικό περιβάλλον, είναι παραμετροποιήσιμο και περιέχει την δυνατότητα παραγωγής στατιστικών στοιχείων για τις υπηρεσίες που είναι διαθέσιμες προς το κοινό. 

Αυτό που δεν υποστηρίζει είναι η δημιουργία πολλαπλών πλάνων που να αντιπροσωπεύουν διαφορετικούς τομείς ή υπαλλήλους (όλα τα ραντεβού φαίνονται σε ένα ημερολόγιο).

Παρατηρήσεις : επί πληρωμή, ικανοποιητικά παραμετροποιήσιμο, δημιουργία στατιστικών

\href{http://www.genbook.com}{www.genbook.com}

\subsection{Web Appointment Scheduling System (Open Source)}
Το WASS είναι μια λύση ανοιχτού κώδικα, η οποία περιέχει τις βασικότερες λειτουργίες διαχείρισης ραντεβού για μια επιχείρηση. Από την εφαρμογή αυτή λείπουν κάποια στοιχεία διαχείρισης και παραμετροποίησης και το γραφικό περιβάλλον του χρήστη χρειάζεται επιπλέον δουλειά. Παρ' όλα αυτά είναι δωρεάν και προτείνεται για οποιαδήποτε μικρή επιχείρηση. Η εφαρμογή υποστηρίζει το iCal της Apple και την δημιουργία πολλών πλάνων.

Παρατηρήσεις : δωρεάν, βασικές λειτουργίες, ανοιχτός κώδικας, υποστήριξη iCal

\href{https://wass.princeton.edu/pages/login.page.php}{www.wass.princeton.edu} |
\href{http://sourceforge.net/projects/wass/}{www.sourceforge.net}

\subsection{Appointment-plus}
Το Appointment-plus είναι από τις πιο οργανωμένες και εμπλουτισμένες εφαρμογές που υπάρχουν σε αυτόν τον τομέα. Έχει εκδόσεις για οποιαδήποτε συσκευή (pc, tablets, smartphones), όμορφο περιβάλλον, υποστήριξη συγχρονισμού δεδομένων με άλλες υπηρεσίες και εφαρμογές (Google, Outlook, iCal κτλ), χρήση από πολλούς υπαλλήλους και πολλά πλάνα. Παρέχει λειτουργία αναμονής για τους πελάτες σε περίπτωση που η επιθυμητή ώρα είναι πιασμένη. Υπάρχει σύστημα email στα οποία ο πελάτης μπορεί να κάνει επιλογές και να τις αποστείλει πίσω στο σύστημα. Δυνατότητα για τροποποίηση της σελίδας που βλέπει ο χρήστης και πώλησης προϊόντων μέσω της εφαρμογής. Η εταιρεία προσφέρει σε κάθε πελάτη έναν βοηθό στον οποίο θα μπορεί να απευθυνθεί για να ρυθμίσει την εφαρμογή. 

Παρατηρήσεις : επί πληρωμή, πλήρως παραμετροποιήσιμο, λειτουργία σε διαφορετικές συσκευές, πολλαπλά πλάνα, υποστήριξη (τηλέφωνο - email), το χρησιμοποιούν μεγάλες εταιρείες, οργανισμοί και πανεπιστημιακά ιδρύματα

\href{http://www.appointment-plus.com/}{www.appointment-plus.com}

\subsection{Acuity Scheduling}
Η εφαρμογή αυτή αν και πιο απλή από την Appointment-plus περιέχει όλες τις βασικές λειτουργίες που θα χρειαστεί μια επιχείρηση για την υλοποίηση ενός συστήματος ραντεβού. Υποστηρίζει την δυνατότητα τροποποίησης της εμφάνισης, δημιουργία πολλών υπαλλήλων και πλάνων, ηλεκτρονικών πληρωμών με πιστωτική κάρτα, εξαγωγή ημερολογίου σε άλλες εφαρμογές (Facebook, Google και Outlook), ιστορικό, διαχείριση πελατών και λειτουργία σε iPhone. Επίσης δίνει την δυνατότητα πώλησης προϊόντων (eshop) για ηλεκτρονικές αγορές.

Παρατηρήσεις : επί πληρωμή (δωρεάν για έναν χρήστη αλλά με περιορισμούς), υποστήριξη προϊόντων, iPhone, εξαγωγή σε Google, Outlook και Facebook

\href{http://www.acuityscheduling.com/}{www.acuityscheduling.com}

\subsection{SetMore}
Το SetMore έχει απλή και όμορφη εμφάνιση και παρέχει ένα πλήρως παραμετροποιήσιμο περιβάλλον. Είναι το μόνο που υποστηρίζει SMS ειδοποιήσεις και αυτό σε beta στάδιο. Η διαδικασία κράτησης ενός ραντεβού χωρίζεται όπως και με τα υπόλοιπα συστήματα, σε έναν οδηγό με 4-5 βήματα στα οποία ο χρήστης επιλέγει σε ποια υπηρεσία και σε ποιόν υπάλληλο θέλει να κλείσει ραντεβού. Είναι πολύ εύκολο στην χρήση και υποστηρίζει plugins για το WordPress και το Facebook.

Παρατηρήσεις : επί πληρωμή, υποστήριξη SMS, λειτουργία με WordPress και Facebook, δεν μπορεί να εξάγει τα δεδομένα του σε άλλη εφαρμογή (Google Calendar, Outlook)

\href{http://www.setmore.com/}{www.setmore.com}

\subsection{Υπόλοιπα συστήματα}
Εκτός των αναφερθέντων υπάρχουν πολλές άλλες εφαρμογές διαθέσιμες προς το κοινό. Μερικές από αυτές αναφέρονται στην παρακάτω λίστα:
\begin{enumerate}
\item SnapAppointments
\item Doodle
\item Bookeo
\item ScheduleOnce
\item BookingBug
\item SetSter
\item Agreedo
\item BookedIn
\item Book'd
\item Schedulista
\end{enumerate}

\section{Σε τι διαφέρει από τις υπόλοιπες η προτεινόμενη λύση}
Το Easy!Appointments έχει ως σκοπό να αυτοματοποιήσει την διαδικασία της κράτησης και διαχείρισης ραντεβού για οποιαδήποτε επιχείρηση. Χρησιμοποιώντας τις δυνατότητες που μας παρέχει το διαδίκτυο, μπορεί να υλοποιηθεί ένα σύστημα το οποίο να έχει την δυνατότητα να οργανώσει τα επαγγελματικά πλάνα πολλών υπαλλήλων ταυτόχρονα, επιφέροντας έτσι όχι μόνο την μείωση του χρόνου που απαιτούσαν οι παλιές μέθοδοι διαχείρισης ραντεβού, αλλά και την αύξηση της παραγωγικότητας της επιχείρησης. Οι πελάτες δεν θα χρειάζεται πλέον να τηλεφωνούν ή να πηγαίνουν στο κατάστημα, αλλά θα μπορούν να βλέπουν τις διαθέσιμες ώρες της επιχείρησης και να κλείνουν το ραντεβού τους την επιθυμητή ημερομηνία και ώρα, μέσω του υπολογιστή και του internet. Αυτό έχει ως αποτέλεσμα την ποιοτικότερη αλλά και αποδοτικότερη εξυπηρέτηση τους. Επιπρόσθετα βελτιώνεται η επικοινωνία και η οργάνωση των συντελεστών της επιχείρησης, παρέχοντας δυνατότητες αρχειοθέτησης και διαχείρισης των δεδομένων που αποθηκεύονται στο σύστημα ανά πάσα στιγμή και σε οποιοδήποτε μέρος. Σε αντίθεση με τα άλλα συστήματα, προσφέρει επιπλέον τα εξής:

\begin{enumerate}
\item {\bf Αυτόνομη Εγκατάσταση :} Η επιχείρηση που θέλει να χρησιμοποιήσει την εφαρμογή θα μπορεί να την εγκαταστήσει στον server της και να την τρέξει μαζί με κάποιο άλλο site, έχοντας έτσι πλήρη πρόσβαση στα δεδομένα και τον κώδικα. Η διαδικασία της εγκατάστασης και παραμετροποίησης θα είναι παρόμοια με άλλα συστήματα (Joomla, WordPress κτλ) και όσο πιο αυτοματοποιημένη γίνεται.

\item {\bf Διαμόρφωση Πρότυπου Πλάνου :} Το σύστημα θα έχει ενσωματωμένη δυνατότητα δημιουργίας πρότυπου πλάνου για τον κάθε πάροχο υπηρεσιών, το οποίο θα αποτελεί την βάση της κάθε εβδομάδας και από εκεί και πέρα ο διαχειριστής θα μπορεί να κάνει αλλαγές. Η επανάληψη του πλάνου καθώς και το από ποιά προτυπα πλάνα θα αποτελείται ένας μήνας θα συμπεριλαμβάνονται στην ρύθμιση της εφαρμογής.

\item {\bf Ρύθμιση Δικαιωμάτων Πάροχων :} Ο διαχειριστής θα έχει την δυνατότητα να ορίζει τα δικαιώματα αλλαγών και ρυθμίσεων που θα έχει στην διάθεσή του ο κάθε πάροχος υπηρεσιών. Έτσι μπορεί να διασφαλιστεί η ασφάλεια των δεδομένων όπου κρίνεται αυτό χρήσιμο, είτε να διευκολυνθεί η διαχείριση των ραντεβού έτσι ώστε να μπορεί ο κάθε πάροχος να διαχειρίζεται τα δικά του.

\item {\bf Υποστήριξη Γραμματείας :} Αν παρόλα αυτά η εταιρεία ορίσει κάποια γραμματέα ως υπεύθυνση των ραντεβού, τότε είναι απαραίτητο να μπορεί να διαχειρίζεται μόνο αυτά και τους πελάτες που είναι καταχωρημένοι στο σύστημα. Το Easy!Appointments υποστηρίζει την δημιουργία χρηστών που αντιπροσωπεύουν αυτόν τον σκοπό.

\item {\bf Αμφίδρομος Συγχρονισμός με το Google Calendar :} Το σύστημα θα υποστηρίζει τον αμφίδρομο συγχρονισμό ραντεβού με το Google Calendar, κάνοντας χρήση του Google Calendar API. Με αυτόν τον τρόπο η διαχείριση των ραντεβού μπορεί να γίνει ακόμα πιο εύκολη, λαμβάνοντας υπόψιν το πόσο δημοφιλής είναι η συγκεκριμένη υπηρεσία της Google.
\end{enumerate}

