%% =============================
%% GENERAL SETTINGS
%% =============================
\documentclass[12pt]{article}

\usepackage[margin=0.7in]{geometry}
\usepackage {graphics}
\usepackage {xltxtra} 
\usepackage {xgreek} 
\usepackage {color}
\usepackage {hyperref}
\hypersetup {colorlinks}


\setmainfont[Mapping=tex-text]{Tahoma} 
\setlength{\parindent}{0cm} 				%% No paragraph indent

\definecolor{darkred}{rgb}{0.5,0,0}
\definecolor{darkgreen}{rgb}{0,0.5,0}
\definecolor{darkblue}{rgb}{0,0,0.5}

\hypersetup{ colorlinks,
linkcolor=darkblue,
filecolor=darkgreen,
urlcolor=darkblue,
citecolor=darkred }

%% =============================
%% DOCUMENT PROPERTIES
%% =============================
\title{{\Huge {\bf Easy!Appointments}} \\[0.3cm] Εισαγωγικό Κεφάλαιο}
\author{Αλέξανδρος Τσελεγγιδης}
\date{Νοέμβριος 2012}

%% =============================
%% DOCUMENT CONTENT
%% =============================
\begin{document}
\maketitle 
\thispagestyle{empty} %% Απομάκρυνση page number από την πρώτη σελίδα
\pagebreak

%% ΕΙΣΑΓΩΓΙΚΟ ΣΗΜΕΙΩΜΑ
\section* {Εισαγωγή} %% Ο αστερίσκος δεν βάζει νούμερο στο section
Το παρόν κεφάλαιο επεξηγεί τον σκοπό ανάπτυξης της εφαρμογής καθώς και τις ανάγκες που καλύπτει σε μια επιχείρηση. Επιπρόσθετα αναφέρονται παρόμοια συστήματα και οι διαφορές που έχουν σε σχέση με το Easy!Appointments. 

%% ΠΟΙΟ ΠΡΟΒΛΗΜΑ ΠΡΟΣΠΑΘΕΙ ΝΑ ΛΥΣΕΙ Η ΕΦΑΡΜΟΓΗ
\section {Ποιο πρόβλημα προσπαθεί να λύσει η εφαρμογή}
Το Easy!Appointments έχει ως σκοπό να αυτοματοποιήσει την διαδικασία της κράτησης και διαχείρισης ραντεβού για οποιαδήποτε επιχείρηση. Χρησιμοποιώντας τις δυνατότητες που μας παρέχει το διαδίκτυο, μπορεί να υλοποιηθεί ένα  σύστημα το οποίο να έχει την δυνατότητα να οργανώσει τα επαγγελματικά πλάνα πολλών υπαλλήλων ταυτόχρονα, επιφέροντας έτσι όχι μόνο την μείωση του χρόνου που απαιτούσαν οι παλιές μέθοδοι διαχείρισης ραντεβού, αλλά και την αύξηση της παραγωγικότητας της επιχείρησης. Οι πελάτες δεν θα χρειάζεται πλέον να τηλεφωνούν ή να πηγαίνουν στο κατάστημα, αλλά θα μπορούν να βλέπουν το πλάνο της επιχείρησης και να κλείνουν το ραντεβού την επιθυμητή ημερομηνία και ώρα, μέσω του υπολογιστή τους. Αυτό έχει ως αποτέλεσμα την ποιοτικότερη αλλά και αποδοτικότερη εξυπηρέτηση τους. Επιπρόσθετα βελτιώνεται η επικοινωνία και η οργάνωση των συντελεστών της επιχείρησης, παρέχοντας δυνατότητες αρχειοθέτησης και διαχείρισης των δεδομένων που αποθηκεύονται στο σύστημα ανά πάσα στιγμή και σε οποιοδήποτε μέρος.

%% ΓΙΑΤΙ ΕΙΝΑΙ ΣΗΜΑΝΤΙΚΟ ΤΟ ΠΡΟΒΛΗΜΑ ΑΥΤΟ
\section {Γιατί ειναι σημαντικό το πρόβλημα αυτό}
Οι απαιτήσεις και η ανταγωνιστικότητα που υπάρχει μεταξύ των επιχειρήσεων στην εποχή μας, απαιτεί την γρήγορη και άμεση διεκπεραίωση διεργασιών και την όσο το δυνατόν καλύτερη οργάνωση τους, για να μπορούν να παρέχουν υπηρεσίες υψηλού επιπέδου με το χαμηλότερο δυνατό κόστος και προσωπικό. Για να επιτύχουν τον σκοπό αυτό οι επιχειρήσεις πρέπει να επιλέξουν τα κατάλληλα εργαλεία οργάνωσης και εξυπηρέτησης των πελατών τους.
\\[0.3cm]
Στην δικιά μας περίπτωση, η κράτηση ενός ραντεβού τις περισσότερες φορές απαιτεί την απασχόληση ενός επιπλέον ατόμου (γραμματέας), το οποίο θα αφιέρωνε αρκετό από το χρόνο του στο να κάνει αυτήν δουλειά. Χρόνος τον οποίο θα μπορούσε να διαθέσει για να ασχοληθεί με άλλες εργασίες ή θέματα. 
\\[0.3cm]
Έπειτα, χρησιμοποιώντας μη ηλεκτρονικά μέσα, η αρχειοθέτηση και η διαχείριση των δεδομένων που κρατούνται από την διαδικασία αυτή, είναι δύσκολη και χρονοβόρα και κατ' επέκταση στις περισσότερες περιπτώσεις ανύπαρκτη. Για παράδειγμα η διαδικασία της αλλαγής ενός ραντεβού στο πλάνο είναι πιο δύσκολη και επιρρεπείς σε σφάλματα. Η πρόσβαση των δεδομένων των κρατήσεων είναι δυνατή μόνο στο γραφείο και μόνο από αυτόν που διαχειρίζεται τα ραντεβού. Αυτό σημαίνει ότι αν για κάποιο λόγο αυτός δεν είναι διαθέσιμος (πχ άδεια ή ασθένεια) η επιχείρηση θα χρειαστεί να εκπληρώσει αυτήν την έλλειψη με κάποιον μη αποδοτικό τρόπο. 
\\[0.3cm]
Τέλος, όταν πρόκειται για μεγάλους οργανισμούς και επιχειρήσεις, το παλιό σύστημα κρατήσεων είναι αδύνατο να λειτουργήσει σωστά. Ακόμα και η χρήση λογισμικού για την καταχώρηση των ραντεβού από τους υπαλλήλους προϋποθέτει την απασχόλησή τους για αυτήν την εργασία, πράγμα που σημαίνει μείωση ανθρώπινου δυναμικού. Στο παράδειγμα των νοσοκομείων η κράτηση ραντεβού για εξέταση με έναν γιατρό είναι μια κουραστική διαδικασία για τον ασθενή, μιας και θα χρειαστεί να περιμένει στο τηλέφωνο αρκετά έως ότου να μπορέσει να πιάσει γραμμή και να κανονίσει το πότε θα εξεταστεί.
\\[0.3cm]
Παρατηρούμε λοιπόν ότι η σημερινή οργάνωση των επιχειρήσεων που λειτουργούν με ραντεβού, θα μπορούσε να βελτιωθεί με την χρήση ενός ηλεκτρονικού συστήματος που θα επίλυε τα προαναφερθέντα προβλήματα και θα πρόσδιδε μεγαλύτερη ευκολία στην εξυπηρέτηση του κοινού. Η προτεινόμενη λύση αποσκοπεί στο να εκπληρώσει αυτά τα κενά και να εντάξει στο ενεργητικό της επιχείρησης ένα δυνατό εργαλείο οργάνωσης.

%% ΠΑΡΟΜΟΙΕΣ ΛΥΣΕΙΣ
\section{Παρόμοιες λύσεις που υπάρχουν ήδη}
\subsection{Genbook}
Το Genbook είναι μια online υπηρεσία που προσφέρει στις επιχειρήσεις την δυνατότητα να εγγραφούν (πληρώνοντας το αντίτιμο) και να χρησιμοποιήσουν την εφαρμογή που τους επιτρέπει να διαχειρίζονται τα ραντεβού. Παρέχει αρκετά φιλικό περιβάλλον, είναι παραμετροποιήσιμο και περιέχει την δυνατότητα παραγωγής στατιστικών στοιχείων για τις υπηρεσίες που είναι διαθέσιμες προς το κοινό. 
\\[0.3cm]
Αυτό που δεν υποστηρίζει είναι η δημιουργία πολλαπλών πλάνων που να αντιπροσωπεύουν διαφορετικούς τομείς ή υπαλλήλους (όλα τα ραντεβού φαίνονται σε ένα ημερολόγιο).
\\[0.3cm]
Παρατηρήσεις : επί πληρωμή, πλήρως παραμετροποιήσιμο, δημιουργία στατιστικών
\\[0.3cm]
\href{http://www.genbook.com}{Website}

\subsection{Web Appointment Scheduling System (WASS)}
Το WASS είναι μια λύση ανοιχτού κώδικα, η οποία περιέχει τις βασικότερες λειτουργίες διαχείρισης ραντεβού για μια επιχείρηση. Από την εφαρμογή αυτή λείπουν κάποια στοιχεία διαχείρισης και παραμετροποίησης και η εμπειρία χρήστη χρειάζεται επιπλέον δουλειά. Παρ' όλα αυτά είναι δωρεάν και προτείνεται για οποιαδήποτε μικρή επιχείρηση. Η εφαρμογή υποστηρίζει το iCal της Apple και την δημιουργία πολλών πλάνων.
\\[0.3cm]
Παρατηρήσεις : δωρεάν, βασικές λειτουργίες, ανοιχτός κώδικας, υποστήριξη iCal
\\[0.3cm]
\href{https://wass.princeton.edu/pages/login.page.php}{Website} |
\href{http://sourceforge.net/projects/wass/}{Source Code}

\subsection{Appointment-plus}
Το Appointment-plus είναι από τις πιο οργανωμένες και εμπλουτισμένες εφαρμογές που υπάρχουν σε αυτόν τον τομέα. Έχει εκδόσεις για οποιαδήποτε συσκευή (pc, tablets, smartphones), όμορφο περιβάλλον, υποστήριξη συγχρονισμού δεδομένων με άλλες υπηρεσίες και εφαρμογές (Google, Outlook, iCal κτλ), χρήση από πολλούς υπαλλήλους και πολλά πλάνα. Παρέχει λειτουργία αναμονής για τους πελάτες σε περίπτωση που η επιθυμητή ώρα είναι πιασμένη. Υπάρχει σύστημα email στα οποία ο πελάτης μπορεί να κάνει επιλογές και να τις αποστείλει πίσω στο σύστημα. Δυνατότητα για τροποποίηση της σελίδας που βλέπει ο χρήστης και πώλησης προϊόντων μέσω της εφαρμογής. Η εταιρεία προσφέρει σε κάθε πελάτη έναν βοηθό στον οποίο θα μπορεί να απευθυνθεί για να ρυθμίσει την εφαρμογή. 
\\[0.3cm]
Παρατηρήσεις : επί πληρωμή, πλήρως παραμετροποιήσιμο, λειτουργία σε διαφορετικές συσκευές, πολλαπλά πλάνα, υποστήριξη (τηλέφωνο - email), το χρησιμοποιούν μεγάλες εταιρείες, οργανισμοί και πανεπιστημιακά ιδρύματα
\\[0.3cm]
\href{http://www.appointment-plus.com/}{Website}

\subsection{Acuity Scheduling}
Η εφαρμογή αυτή αν και πιο απλή από την Appointment-plus περιέχει όλες τις βασικές λειτουργίες που θα χρειαστεί μια επιχείρηση για την υλοποίηση ενός συστήματος ραντεβού. Υποστηρίζει την δυνατότητα τροποποίησης της εμφάνισης, δημιουργία πολλών υπαλλήλων και πλάνων, ηλεκτρονικών πληρωμών με πιστωτική κάρτα, εξαγωγή ημερολογίου σε άλλες εφαρμογές (Facebook, Google και Outlook), ιστορικό, διαχείριση πελατών και λειτουργία σε iPhone. Επίσης δίνει την δυνατότητα πώλησης προϊόντων (eshop) για ηλεκτρονικές αγορές.
\\[0.3cm]
Παρατηρήσεις : επί πληρωμή (δωρεάν για έναν χρήστη αλλά με περιορισμούς), υποστήριξη προϊόντων, iPhone, εξαγωγή σε Google, Outlook και Facebook
\\[0.3cm]
\href{http://www.acuityscheduling.com/}{Website}

\subsection{SetMore}
Το SetMore έχει απλή και όμορφη εμφάνιση και παρέχει ένα πλήρως παραμετροποιήσιμο περιβάλλον. Είναι το μόνο που υποστηρίζει sms ειδοποιήσεις και αυτό σε beta στάδιο. Η διαδικασία κράτησης ενός ραντεβού χωρίζεται όπως και με τα υπόλοιπα, σε έναν οδηγό με 4-5 βήματα στα οποία ο χρήστης επιλέγει ποια υπηρεσία και σε ποιόν υπάλληλο θέλει να κλείσει ραντεβού. Είναι πολύ εύκολο στην χρήση και υποστηρίζει το WordPress και το Facebook.
\\[0.3cm]
Παρατηρήσεις : επί πληρωμή, υποστήριξη sms, λειτουργία με WordPress και Facebook, δεν μπορεί να εξάγει τα δεδομένα του σε άλλη εφαρμογή (Google Calendar, Outlook)
\\[0.3cm]
\href{http://www.setmore.com/}{Website}

\subsection{Υπόλοιπα συστήματα}
Εκτός των αναφερθέντων υπάρχουν πολλές άλλες εφαρμογές διαθέσιμες προς το κοινό. Μερικές από αυτές αναφέρονται στην παρακάτω λίστα:
\begin{enumerate}
\item SnapAppointments
\item Doodle
\item Bookeo
\item ScheduleOnce
\item BookingBug
\item SetSter
\item Agreedo
\item BookedIn
\item Book'd
\item Schedulista
\end{enumerate}

%% ΣΕ ΤΙ ΔΙΑΦΕΡΕΙ ΑΠΟ ΤΙΣ ΥΠΟΛΟΙΠΕΣ Η ΠΡΟΤΕΙΝΟΜΕΝΗ ΛΥΣΗ
\section{Σε τι διαφέρει από τις υπόλοιπες η προτεινόμενη λύση}
Το Easy!Appointments υλοποιεί σε γενικές γραμμές ότι και όλα τα υπόλοιπα προγράμματα (εκτός λίγων εξαιρέσεων), προσφέροντας επιπλέον τα εξής:

\begin{enumerate}
\item {\bf Αυτόνομη Εγκατάσταση :} Η επιχείρηση που θέλει να χρησιμοποιήσει την εφαρμογή θα μπορεί να την εγκαταστήσει στον server της και να την τρέξει μαζί με κάποιο άλλο site, έχοντας έτσι πλήρη πρόσβαση στα δεδομένα και τον κώδικα. Η διαδικασία της εγκατάστασης και παραμετροποίησης θα είναι παρόμοια με άλλα συστήματα (Joomla, WordPress κτλ) και όσο πιο αυτοματοποιημένη γίνεται.

\item {\bf Προεπιλεγμένα Προφίλ Ρυθμίσεων :} To Easy!Appointments θα έρχεται μαζί με κάποια προεπιλεγμένα προφίλ ρυθμίσεων ανάλογα με το είδος της επιχείρησης, έτσι ώστε να είναι πιο εύκολο στην ρύθμισή του. Τα προφίλ αυτά θα μπορούν να αποθηκεύονται και εξωτερικά για να μπορούν να διαμοιράζονται και στην κοινότητα των χρηστών της εφαρμογής.

\item {\bf Πρότυπο Πλάνο :} Το σύστημα θα έχει ενσωματωμένη δυνατότητα δημιουργίας πρότυπου πλάνου το οποίο θα αποτελεί την βάση της κάθε εβδομάδας και από εκεί και πέρα ο διαχειριστής θα μπορεί να κάνει αλλαγές. Η επανάληψη του πλάνου καθώς και το από ποιά προτυπα πλάνα θα αποτελείται ένας μήνας θα συμπεριλαμβάνονται στην ρύθμιση της εφαρμογής.

\item {\bf Ειδοποιήσεις SMS :} Τόσο οι διαχειριστές, όσο και ο πελάτες θα μπορούν να λαμβάνουν  ειδοποιήσεις με sms, εφόσον αυτή η υπηρεσία έχει ρυθμιστεί όπως πρέπει. Με αυτόν τον τρόπο δεν θα είναι πάντα απαραίτητο να υπάρχει σύνδεση με το internet. Επίσης ο διαχειριστής θα μπορεί να στέλνει ένα sms στο server και να λαμβάνει τα ραντεβού της ημέρας που τον ενδιαφέρει.

\end{enumerate}
\end{document}